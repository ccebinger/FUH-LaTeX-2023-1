\documentclass[12pt,ngerman, parskip=half]{scrartcl}

\usepackage{babel}
\usepackage{blindtext}
\usepackage{microtype}

\author{Uwe Ziegenhagen}
\title{Mein erstes LaTeX-Dokument}
\date{Entenhausen, den \today}
% \date{} für kein Datum

\usepackage{xcolor}
\newcommand{\person}[1]{\textcolor{magenta}{\textsc{#1}}}

\usepackage{hyperref}
\hypersetup{
    bookmarks=true,                     % show bookmarks bar
    unicode=false,                      % non - Latin characters in Acrobat’s bookmarks
    pdftoolbar=true,                        % show Acrobat’s toolbar
    pdfmenubar=true,                        % show Acrobat’s menu
    pdffitwindow=false,                 % window fit to page when opened
    pdfstartview={FitH},                    % fits the width of the page to the window
    pdftitle={My title},                        % title
    pdfauthor={Author},                 % author
    pdfsubject={Subject},                   % subject of the document
    pdfcreator={Creator},                   % creator of the document
    pdfproducer={Producer},             % producer of the document
    pdfkeywords={keyword1, key2, key3},   % list of keywords
    pdfnewwindow=true,                  % links in new window
    colorlinks=true,                        % false: boxed links; true: colored links
    linkcolor=blue,                          % color of internal links
    filecolor=cyan,                     % color of file links
    citecolor=green,                     % color of file links
    urlcolor=magenta                        % color of external links
}


\begin{document}
\maketitle

\tableofcontents

\section{Einleitung}\label{sec:einleitung}


\section*{Hallo} % Mit Sternchen => taucht nicht im Inhaltsverzeichnis auf

\blindtext

Siehe Abschnitt \ref{sec:einleitung} auf Seite \pageref{sec:einleitung}.

Ich bin ein Text, der in der Standardschriftgröße gesetzt wird. Ich bin ein Text, der in der Standardschriftgröße gesetzt wird.

Ich bin ein Text, der in der \textbf{Standardschriftgröße} gesetzt wird. \person{Albert Einstein} war ein \textit{bedeutender} \textsl{bedeutender}  Physiker des \textit{\textbf{zwanzigsten}} Jahrhunderts.

\textcolor{red}{Uwe Ziegenhagen}

\person{Stephen Hawking}  war ein \emph{cooler} Typ.

\section[Literatur]{Literaturüberblick über wichtige Quellen des ausgehenden Jahrhunderts in diversen Teilen der Welt}


\subsection{Literatur vor 1918}
\subsubsection{Europa}

\blindtext[2]

\subsubsection{Asien}

\blindtext[2]


\section{Analyse}

\blindtext[2]

\section{Fazit}

\blindtext[2]



\end{document}




