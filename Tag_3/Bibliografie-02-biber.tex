%!TEX TS-program = Arara
% arara: pdflatex: {shell: yes}
% arara: biber
% arara: pdflatex: {shell: yes}
\documentclass[12pt,ngerman]{scrartcl}

\usepackage{babel}
\usepackage{blindtext}

\usepackage[style=nature ,backref=true,backend=biber]{biblatex}
\usepackage[babel,german=quotes]{csquotes}

\addbibresource{ErsteBibliothek.bib}	

\usepackage{hyperref}
\hypersetup{
    bookmarks=true,                     % show bookmarks bar
    unicode=false,                      % non - Latin characters in Acrobat’s bookmarks
    pdftoolbar=true,                        % show Acrobat’s toolbar
    pdfmenubar=true,                        % show Acrobat’s menu
    pdffitwindow=false,                 % window fit to page when opened
    pdfstartview={FitH},                    % fits the width of the page to the window
    pdftitle={My title},                        % title
    pdfauthor={Author},                 % author
    pdfsubject={Subject},                   % subject of the document
    pdfcreator={Creator},                   % creator of the document
    pdfproducer={Producer},             % producer of the document
    pdfkeywords={keyword1, key2, key3},   % list of keywords
    pdfnewwindow=true,                  % links in new window
    colorlinks=true,                        % false: boxed links; true: colored links
    linkcolor=blue,                          % color of internal links
    filecolor=cyan,                     % color of file links
    citecolor=blue,                     % color of file links
    urlcolor=magenta                        % color of external links
}
\begin{document}

\blindtext

Dies hier ist ein Blindtext zum Testen von Textausgaben, siehe \cite{Ziegenhagen2023}. Wer diesen Text liest, ist selbst schuld. Der Text gibt lediglich den Grauwert der Schrift an. Ist das wirklich so? Ist es gleichgültig, ob ich schreibe: „Dies ist ein Blindtext“ oder „Huardest gefburn“? Kjift – mitnichten! Ein \cite{Knuth1982} Blindtext bietet mir wichtige Informationen. An ihm messe ich die Lesbarkeit einer Schrift, ihre Anmutung, wie harmonisch die Figuren zueinander stehen und prüfe, wie breit oder schmal sie läuft. Ein Blindtext sollte möglichst \footcite{Ziegenhagen2022} viele verschiedene Buchstaben enthalten und in der Originalsprache gesetzt sein. Er muss keinen Sinn ergeben, sollte aber lesbar sein. Fremdsprachige Texte wie „Lorem ipsum“, siehe \cite{Knuth1982}, dienen nicht dem eigentlichen Zweck, da sie eine falsche Anmutung vermitteln.

\clearpage

\parencite{Knuth1982}

\clearpage

Ein Blindtext sollte möglichst viele verschiedene Buchstaben\footnote{Ein Blindtext sollte möglichst viele verschiedene Buchstaben enthalten und in der Originalsprache gesetzt sein. Er muss keinen Sinn ergeben, sollte aber lesbar sein.} enthalten und in der Originalsprache gesetzt sein. Er muss keinen Sinn ergeben, sollte aber lesbar sein. \footcite{Knuth1982}

\clearpage

\citeauthor{Ziegenhagen2023} hat in seinem im Jahr \citeyear{Ziegenhagen2023} erschienenen Artikel \citetitle{Ziegenhagen2023} bewiesen, dass die Erde rund ist.

\printbibliography[title={Artikel},type=article]

\printbibliography[title={Bücher},type=book]

\printbibliography[title={Internetreferenzen},type=online]


\end{document}