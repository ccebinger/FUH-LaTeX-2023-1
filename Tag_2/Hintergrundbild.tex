\documentclass[12pt,ngerman]{beamer}

\usepackage{babel}
\usepackage{blindtext}

\usepackage{transparent}

\author{Uwe Ziegenhagen}
\title{Meine erste Präsentation}
\institute{Dante e.V. Heidelberg}

\usebackgroundtemplate{\transparent{0.8}\includegraphics[width=\paperwidth]{Bilder/Katze}}

\usetheme{Warsaw}
\usecolortheme{wolverine}

\definecolor{fuh}{rgb}{0.4,0.25,1}
\definecolor{FUH}{RGB}{150,75,100}

\begin{document}

\begin{frame}

\maketitle

\end{frame}

\begin{frame}

\tableofcontents

\end{frame}

\usebackgroundtemplate{}

\begin{frame}
\frametitle{Einleitung}
\framesubtitle{Literatur}

\begin{itemize}
	\item Mein
	\item e
	\item Auf
	\item zählung
	\item als itemize
	\item items
\end{itemize}

\end{frame}

\section{Einleitung}

\begin{frame}
\frametitle{Einleitung}
\framesubtitle{Literatur}

\begin{enumerate}
	\item<1-> Mein
	\item<2> e
	\item<1,3,4> Auf
	\item<-3> zählung
	\item als itemize
	\item items
\end{enumerate}

\end{frame}

\begin{frame}
\frametitle{Einleitung}
\framesubtitle{Literatur}

\begin{description}
	\item[Apfel] Eine \textcolor{fuh}{Frucht}
	\item[Birne] Auch eine \textcolor{FUH}{Frucht}
\end{description}

\[ a^2 + b^2 = c^2 \]

\end{frame}

\section{Hauptteil}

\begin{frame}
\frametitle{Blindtext}

\blindtext

\end{frame}

\begin{frame}
\frametitle{Mehrspaltig}

\begin{columns}
\begin{column}{0.33\textwidth}
\tiny\blindtext
\end{column}
\begin{column}{0.33\textwidth}
\tiny\blindtext
\end{column}
\begin{column}{0.33\textwidth}
\tiny\blindtext
\end{column}
\end{columns}


\end{frame}



\end{document}